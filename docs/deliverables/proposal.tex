\documentclass[12pt,draft]{report}
\begin{document}

\title{Capstone Project Proposal}

\author{\begin{tabular}{rl}
	\textbf{Project Number:} & S16-031 \\
	\textbf{Project Title:} & Distributed Machine Learning Using Raspberry Pi’s \\
	\textbf{Project Term:} & Spring 2016 \\
	\textbf{Students:} & Nikhil Shenoy \\ & Revan Sopher \\
	\textbf{Supervisor:} & Dr. Anand D. Sarwate \\
\end{tabular}}

% \author{
% 	\textbf{Project Number:} \\ S16-031 \\
% 	\textbf{Project Title:} \\ Distributed Machine Learning Cluster with Raspberry Pi’s \\
% 	\textbf{Project Term:} \\ Spring 2016 \\
% 	\textbf{Students:} \\ Nikhil Shenoy \quad Revan Sopher \\
% 	\textbf{Professor:} \\ Dr. Anand D. Sarwate \\
% }
\date{\today}
\maketitle

\newpage

\section*{Description:}

Distributed computing refers to the use of multiple networked computers to perform data analysis at scale, the cornerstone of the “big data” trend. Traditional efforts focus on using powerful consumer-grade workstations housed in a data center, and analyze data from applications such as social activity on Facebook to particle collisions at the Large Hadron Collider. The parallel and distributed computing capabilities of these distributed systems have allowed organizations to make huge strides in their endeavors. However, constructing such a system requires an immense amount of resources and man power. We seek to bring the capabilities of large scale computing clusters to the hands of every day users by creating a flexible, cost-efficient wireless sensor network designed for INSERT APPLICATION HERE.

Describe the application here

Advantages of using Raspberry Pis/smaller networks

Problems we want to solve and anticipate running into
\begin{itemize}
	\item Power consumption
	\item Communication between nodes
	\item Gathering of and interpretation of results
	\item Portability and extensibility
\end{itemize}

\section*{Matieral from original draft}
but we propose the construction of a distributed computing cluster from multiple Raspberry Pi’s -- inexpensive, low power devices suitable for embedded use. This novel arrangement, generally defined as a sensor network, creates a new field of challenges, notably in working with the constraints of considerably weaker hardware, but allows for exciting real world scenarios -- for example, distributed fault detection by installing the devices on highway infrastructure. Tackling such scenarios is infeasible for 



Such an application would be ruled out by a traditional datacenter cluster -- the workstations are too large and power-hungry to be deployed in the field, but power costs of maintaining a constant uplink from the sensors back to the data center are also prohibitive.

Several instances of sensor networks have been designed and implemented, with remarkable results. The company ShotSpotter has exploited the advantages of ad hoc sensor networks in order to monitor gun violence in urban areas; by deploying a series of microphones on the rooftops of buildings, the company can monitor entire neighborhoods for suspicious sounds. The system then collects the audio data recorded by each microphone and decides whether a gun was fired. We aim to develop a similar system using Raspberry Pi’s, with the intention that those with relatively little knowledge of distributed computing and sensor networks can obtain a cheap, portable, and efficient system with which they can analyze data. 

Preliminary work suggests that deployment infrastructure to update the devices will be the first challenge. We expect to use Docker to containerize the application, and resin.io to push updates to the fleet. To abstract away the inter-device networking, we will base our data analysis on Apache Spark.

\end{document}
